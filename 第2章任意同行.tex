\documentclass[11pt]{jsarticle}

\usepackage[sect]{kian}
\usepackage{otf}
\usepackage{fancybox}
\usepackage{ascmac}
\usepackage[noalphabet]{pxchfon}  
\setminchofont{A-OTF-RyuminPro-Light.otf}
\setgothicfont{A-OTF-FutoGoB101Pr6N-Bold.otf}

\setlength{\marginparwidth}{40mm}



\title{\vspace{-30mm}{\textgt{\Large{\fbox{2} 任意同行 }}}}
\date{\vspace{-15mm}}


\begin{document}

\maketitle
\begin{itembox}[l]{以下の行為の適法性を検討しなさい。}
	KらがXの車両の周囲に警察車両を停める、窓から手を差し入れてエンジンキーを抜き取るなどして、
	約4時間にわたってXを留め置いた行為
\end{itembox}

\sectionB{}
	Kらの行った本件職務質問における留め置きは強制処分に当たるか。
	また、当たらないとしても任意処分として適法か。
	
	警職法2条3項が、刑訴法の規定によらない限り、身柄を拘束されることはない旨規定していることから問題となる。
\sectionB{}
	そもそも職務質問とは、不審事由のある者に対し停止させて質問することをいう。
	職務質問は行政警察活動としての性質を有し、警職法2条3項により任意処分であることが明らかにされている。
	したがって、本件職務質問が強制処分に当たれば、令状主義違反として直ちに違法となる。
	そこで、本件職務質問が強制処分に当たらないかがまず問題となる。

	強制処分には強制処分法定主義、令状主義の規律が及ぶ。
	そこで、強制処分とは、このような厳格な要件・手続によって統制を受けるにふさわしい処分であることを要するというべきである。
	具体的には、個人の意思を制圧し、身体、住居、財産等に制約を加えて強制的に捜査目的を実現する行為など、特別の根拠規定がなければ許容することが相当でない手段を用いた処分をいうものと解すべきである。
	
	本件において、Xらが乗っていた高級自動車に多数の損傷があること、警察官であるKらを見ないようにしていること、顔色から薬物常用者のようであること、
	覚せい剤事犯の犯罪歴があること、注射痕らしい痕跡等から、Xらには覚せい剤使用の強い嫌疑が認められる。
	にも関わらず、Xらは任意同行に素直に応じず、車に乗り込んで現場から離れる可能性まで生じた。
	このようなXらに対して、Kらは任意同行と尿の提出をするように説得をしたに過ぎない。
	また、エンジンキーを抜き取る行為は身体的接触を伴うものではなく、
	さらに、Xらには携帯電話で外部と連絡を取ったり、化粧を直したりするなど、行動の自由はある程度認められていた。
	
	以上のことからすると、Kらの行為は、強制処分とまではいえない。

\sectionB{}
	では、任意処分として適法か。任意処分であっても何らかの法益を侵害するから、状況のいかんを問わずに許容されると解するのは相当ではなく、
	比例原則(警職法2条1項、同条2項)のもと
	必要性、緊急性なども考慮した上、具体的状況の下で相当と認められる限度において許容されると解する。

	\sectionC{}
		Xらには、高級自動車であるにも関わらず車体に多数の損傷があること、
		警察官を見ないようにしていたこと、
		顔色が悪く薬物常用者のようであることなどの不審事由が認められるから、車を停止させて職務質問を開始したこと自体は適法である。
		
	\sectionC{}
		次に、エンジンキーを抜き取り、約4時間にわたってXらを留め置いた行為は任意処分として適法か。
		
		Kらには、Xらの不審事由を解明するという目的が認められる。では、当該目的のためにXらを留め置く必要があったといえるか。
		Xの前科や注射痕などから当初の不審事由は、薬物事犯という重大な犯罪に関する相当の嫌疑にまで高まっており、
		また、Xは自動車に乗り込んだり、クラクションを鳴らしたりしていることから、自動車を発進して現場から立ち去る可能性が認められ、
		これを阻止するため、エンジンキーを抜き取る、自動車が発進できないように周囲に警察車両を停めるなどの措置をとる必要性・緊急性が認められる。
		
		もっとも、XらはKらの説得には応じない態度を崩さなかったにも関わらず、複数の警察官が取り囲み、エンジンキーも返却されない場合、
		事実上車内にとどまらざるを得ない状態であり、この状態で約4時間という長時間にわたり留め置かれることはXらの移動の自由を強く制約するものといえるから、
		相当性を欠くといえる。
		
		したがって、Kらの行為は、相当性を欠く任意処分として違法である。
		
		
		
		
		















\end{document}