\documentclass[11pt]{jsarticle}

\usepackage[sect]{kian}
\usepackage{otf}
\usepackage{fancybox}
\usepackage{ascmac}
\usepackage[noalphabet]{pxchfon}  
\setminchofont{A-OTF-RyuminPro-Light.otf}
\setgothicfont{A-OTF-FutoGoB101Pr6N-Bold.otf}

\setlength{\marginparwidth}{40mm}



\title{\vspace{-30mm}{\textgt{\Large{\fbox{1} 職務質問 }}}}
\date{\vspace{-15mm}}


\begin{document}

\maketitle
\begin{itembox}[l]{以下の行為の適法性を検討しなさい。}
	\begin {enumerate}
		\item Kが自転車を横にしてXの進路を塞ぐように停め、Xに「どうして逃げた」と声をかけた行為
		\item Kが警棒をXの面前で振り上げて見せて「おとなしくしろ」といった行為
		\item KがXのポーチのチャックを開け、中に手を差し入れ、中を探り、白色結晶入りのビニール袋を引っ張り出した行為
	\end{enumerate}
\end{itembox}

\sectionA{行為1について}
	\sectionB{}
		Kは自転車を停めてXの進路を塞ぐことでXを「停止させて質問」しているが、この段階では特定の犯罪の嫌疑は生じていないから、
		Kの行為は行政警察活動としての性質を有する。
		
		
		\sectionC{}
			したがって、Kの質問は警職法2条1項に基づく職務質問であるから、同条項の要件を満たしているかを検討する。Xは、一見して日本人とは思われない風貌であり、「外国人から覚せい剤を購入できる場所」として紹介されている路地において、
			警察官Kの姿を見るや否や不自然な回避行動をとっており、
			「何らかの犯罪を犯し、若しくは犯そうとしていると疑うに足りる相当な事由」(不審事由)が認められる。
			
		\sectionC{}
			Kは、Xの進路を塞ぐようにしてXを「停止」させている。警職法2条2項は不審事由のある者を「停止させて」「質問することができる」と規定していることから、「停止させて」の意義が問題となる。
			
			ここで、「停止させて」という表現は、停止についてある程度の実力行使を認めている根拠と読めるし、
			停止させないことには質問もできないから合理性も認められる。
			もっとも、警職法2条3項は行政警察活動において「強制手段」を用いることを禁止しており、
			ある程度の実力の行使は任意処分の枠の中で理解されるから、
			拒否の自由を残さないような実力の行使は許されないと解すべきである。
			
			また、任意手段であっても、実力の行使による一定の権利侵害の可能性が認められるから、
			警職法1条1項および2項の比例原則により、
			当該手段を用いる必要性・緊急性と権利侵害の程度等を衡量し、
			具体的状況の下で相当と認められる限度でのみ許容されると解される。
			
		\sectionC{}
			本問において、Kが行使した有形力はXの進路を塞ぐという軽度のものであって、強制手段を用いたとまではいえない。
		
			そこで次に、任意処分として許容されるかを検討する。
			Xには前述のような不審事由があり、Kを見るなり向きを変えて走り出しているから、
			職務質問を行うためにXを停止させる必要性と緊急性が認められる。
			そして、走って逃げているXを停止させる手段として、自転車で進路を塞ぐ行為よりも
			侵害性の低い他の手段があったともいえない。
		
	\sectionB{}
		以上より、Kの行為は任意手段として相当性を有するから、適法である。
		
			
\sectionA{行為2について}
	KがXを一度停止させた後、走り出そうとしたXに対して警棒を振り上げて「おとなしくしろ」といった行為は、
	警職法2条3項が禁じる「強制手段」に当たらないか。
	
	武器を持っていないXに対して警棒を振り上げることによって、強度の有形力の行使を示唆して
	Xを停止させる行為は、Xが停止しない場合、警棒によって暴行を加えられるというおそれを惹起するものであって、
	Xの移動の自由という重要な権利を制約する上で、拒否の自由を残さないような態様であったといえる。
	
	したがって、Kの行為は任意処分の枠を超え、強制の手段を用いたと評価されるから、
	警職法2条3項に反し、違法である。

\sectionA{行為3について}
	\sectionB{}
		Kの「ポーチの中を見せてくれ」という要求に対し、Xは何も答えていないが、
		ポーチを抱きかかえる腕を強めているから、Kの行為はXの権利を侵害するものであり、少なくともXの黙示の意思に反しているといえる。
		
		ここで、Kの行為はいわゆる所持品検査に当たるが、
		所持品検査について警職法にはこれを許容する明文の規定は存在しない。
		しかし、侵害留保原則により権利侵害を伴う所持品検査をするには法的根拠が必要である。
		そこで、職務質問に伴う所持品検査の法的根拠をどこに求めるか、その許容性はどの程度かが問題となる。
		
		\sectionC{法的根拠}
			所持品検査は、職務質問と密接な関連性を有し、
			職務質問の効果をあげるうえでの必要性及び有効性が認められるから、
			警職法2条1項の職務質問の付随行為として行うことができる場合があると解される。
		\sectionC{所持品検査の限界}
			もっとも、\UTF{2460}所持品検査は、任意手段である職務質問の付随行為として許容されるのであるから、
			対象者の承諾を得て行うのが原則であるが、
			承諾のない所持品検査であっても「捜索に至らない程度の行為は、強制にわたらない限り」許容される場合がある。

			\UTF{2461}そして、任意手段であっても、所持品検査により憲法35条が保障する権利が害されるから、
			所持品検査の必要性・緊急性と権利侵害の程度等を衡量し、具体的状況の下で相当と認められる限度でのみ許容されると解される。
		\sectionC{}
			本問において、所持品検査の対象はXの持っていたポーチに限定されており、
			犯罪の手がかりとなるものを探索するような「捜索」に類するものではなく、
			Xを押さえつけるなどの有形力も行使されていないから、禁止される強制手段とまではいえない。
			
			そこで次に、任意手段として許容されるかを検討する。
			Kの行った所持品検査は、薬物犯罪が行われている疑いのある場所にいた、Xの不審事由を解明するという目的が認められる。
			
			では、当該目的のために、本件の所持品検査は必要であったといえるか。
			薬物犯罪という重大な事件において、当該犯罪が起きていると目される場所において、
			風貌やKを発見するなり走り出す、携帯していたポーチの中身を見せないといった、
			犯罪と結びつきうる不審事由のあるXに対して、
			その不審事由の根拠でもある所持品の内容を確認する必要が肯定できる。
			
			そして、薬物を所持していた場合、水に流すなどして証拠隠滅も容易であるから、
			緊急に所持品検査を行う必要も認められる。
			
			本件では、所持品検査の態様として、ポーチを開けて中を一別する態様のもの、手を入れて内容物を取り出す態様、
			ポーチをひっくり返して内容物を全て取り出す態様等が想定できるが、
			Xが任意提出を拒み、ポーチを強く抱きかかえている状況において、ポーチを開けて中を一瞥することによって
			内容物を確認することは困難といえるから、
			隙間から手を入れて内容物を取り出す行為は、目的達成のために必要な手段の中で必要最小限度の手段といえる。
			
	\sectionB{}
		以上より、Kの行為は具体的状況の下で相当といえるから、適法である。




\begin{flushright}
	以上
\end{flushright}
	
\end{document}








