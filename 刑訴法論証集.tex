\documentclass[fontsize=11bp]{jlreq}
\usepackage{fancybox}
\usepackage{luatexja-fontspec}
\usepackage{luatexja-otf}
\setmainfont{PRyuminKSpProN-Light}
\setsansfont{PA1GothicStdN-Medium}
\setmainjfont{PRyuminKSpProN-Light}
\setsansjfont{PA1GothicStdN-Medium}

\begin{document}

\part{捜査法}
	\section{捜査の端緒}
		\subsection{職務質問}
			職務質問とは、不審事由のある者を「停止させて」「質問する」ことをいう(警職法2条1項)。
			職務質問の法的性質は任意処分であるから(警職法2条3項)、職務質問が強制処分に当たれば直ちに違法である。
			
			\subsubsection{強制処分の意義}
			強制処分には強制処分法定主義、令状主義の規律が及ぶ。
			そこで、強制処分とは、このような厳格な要件・手続によって統制を受けるにふさわしい処分であることを要するというべきである。
			具体的には、個人の意思に反して、身体、住居、財産等の重要な権利・利益に制約を加えて強制的に捜査目的を実現する行為など、
			特別の根拠規定がなければ許容することが相当でない手段を用いた処分をいうものと解すべきである。
			
			\subsubsection{任意処分の限界}
			任意処分であっても、一定の権利侵害の可能性が認められるから、情況のいかんを問わず常に許容されると解するのは妥当でなく、
			警職法1条1項および2項の比例原則により、
			当該手段を用いる必要性・緊急性と権利侵害の程度等を衡量し、
			具体的状況の下で相当と認められる限度でのみ許容されると解される。
			
			\subsection{職務質問のための停止}
			職務質問は任意処分であるが、有形力の行使はいかなる場合においても許容されないと解すると、警職法1条1項の目的を達成することができない。
			また、同法2条1項は「停止させて」質問することができると規定しており、有形力の行使を許容していると読める。
			
			したがって、一定程度の有形力の行使も許されると解する。もっとも、有形力の行使はあくまでも任意処分の枠の中で理解されるから、
			拒否の事由を残さないような実力の行使は許されないと解すべきである。
			
			また、任意手段であっても何らかの法益を侵害する可能性が認められるから、比例原則(警職法1条1項および2項)により、当該処分を行う必要性と緊急性を比較衡量した上、
			具体的状況の下で相当と認められる範囲でのみ許されると解する。
			
			\subsection{職務質問に伴う所持品検査}
			いわゆる所持品検査について警職法にはこれを許容する明文の規定は存在しない。
			しかし、侵害留保原則により権利侵害を伴う所持品検査をするには法的根拠が必要である。
			そこで、職務質問に伴う所持品検査の法的根拠をどこに求めるか、その許容性はどの程度かが問題となる。
			
			所持品検査は、職務質問と密接な関連性を有し、職務質問の効果をあげるうえでの必要性及び有効性が認められるから、
			警職法2条1項の職務質問の付随行為として行うことができる場合があると解される。
			
			\UTF{2460}所持品検査は、任意手段である職務質問の付随行為として許容されるのであるから、
			対象者の承諾を得て行うのが原則であるが、承諾のない所持品検査も「捜索に至らない程度の行為は、強制にわたらない限り」許容される場合がある。
			
			\UTF{2461}そして、任意手段であっても、所持品検査により憲法35条が保障する権利が害されるから、
			その限界は、所持品検査の必要性・緊急性と権利侵害の程度等を衡量し、具体的状況の下で相当と認められる限度でのみ許容される。
		
	
	\section{逮捕・勾留}
		\subsection{現行犯逮捕の要件}
			現行犯逮捕(刑訴法212条1項)を行うためには、
			\UTF{2460}犯罪と逮捕との時間的場所的接着性が認められることが必要である。
			また、現行犯逮捕が令状主義(憲法33条、刑訴法199条1項)の例外とされている趣旨から、
			\UTF{2461}逮捕者にとって犯罪と犯人が明白であることも要する。
			さらに、必要性のない身体拘束は正当化できないから、通常逮捕と同様、
			\UTF{2462}逮捕の必要性も要件となる。
		\subsection{準現行犯逮捕の要件}
			まず、被逮捕者が刑訴法212条2項各号のいずれかの要件に当たることが必要である。
			
			次に、Xが「罪を行ってから間がないと明らかに認められる」(同項柱書)か。
			ここで、「間がない」とは、犯罪行為と逮捕との時間的場所的接着性が認められることをいう。
			また、そもそも準現行犯人を現行犯人と見なして、無令状で逮捕できるのは、
			犯人であることが明らかであって、誤った逮捕などの人権侵害のおそれが小さいからである。
			そこで「明らか」とは、犯罪と犯人が逮捕者にとって明白であることをいうと解する。
			明白性の判断に当たっては、準現行犯逮捕の場合、ある程度の時間的・場所的な隔たりがあることは前提とされているから、
			逮捕者が直接知覚した事実のみならず、
			共犯者の供述や、既に得ている捜査情報等も補助的な判断資料として用いることが許される。
			
			また、準現行犯逮捕の場合、通常逮捕(刑訴法199条2項ただし書き)のように逮捕の必要性に関する明文の規定は存在しないが、
			身体拘束という重大な権利制約を伴う強制処分であるため、
			準現行犯逮捕も「明らかに逮捕の必要性がない」ときは許されないと解される。
			
		\subsection{再逮捕・再勾留禁止の原則、重複逮捕・勾留禁止の原則}
			\subsubsection{両原則の定義}
			
			再逮捕・再勾留禁止の原則とは、同一の被疑事実を基礎とする逮捕・勾留は1回しか許されないという原則をいう。
			
			重複逮捕・勾留禁止の原則とは、1つの被疑事実を複数に分割しその事実ごとに同時に複数の逮捕・勾留をすることは許されないという原則をいい、
			「1つの被疑事実」とは、実体法上一罪を意味する。
			実体法上一罪には1つの刑罰しか認められないから、刑事手続においても実体法上一罪を構成する事実を1つとして扱うべきであるからである。
			\subsubsection{思考プロセス}
			
			本件の逮捕・勾留は、その実体的要件が認められるから、手続を遵守していれば、
			刑訴法の明文の規定には反しない。
			
			しかし、逮捕・勾留は、原則として被疑事実ごとに一回しか許されない(一罪一逮捕一勾留の原則)。
			同一の被疑事実についても、逮捕・勾留を繰り返すことができるとすると、起訴前の身体拘束期間を制限した法の趣旨が没却されるからである。
			
			ここで、実体法上一罪を構成する被疑事実については1つの刑罰権しか認められないから、
			刑事手続上も実体法上一罪である被疑事実について同時に捜査する義務(同時処理義務)が課される。
			(また、実体法上一罪の関係になくとも、実質的に同一といえる関係にある被疑事実について
			\footnote{社会的事実としては、一連一体の事実であって、関係者も同一であり、必要とされる捜査の内容も大半が共通することなどを考慮する。}、
			同時処理可能性が肯定されれば、同時処理義務が発生する場合がある。)
			
			(被疑事実の同一性、同時処理可能性を確認して、同時処理義務を肯定したら)
			
			同時処理義務の認められる被疑事実につき、再度逮捕・勾留が行なわれた場合、再逮捕・再勾留禁止の原則に反し、直ちに違法となるか。
			
			捜査は流動的であるから、再度の身体拘束をする必要性が生じることは否定できない。
			また、刑訴法199条3項は、再逮捕がありうることを前提としていると解される。
			ここで、同原則の趣旨は、身体拘束の不当な蒸し返しを禁ずる点にあるから、それに当たらなければ、例外的に許容してよいと解される。
			
			したがって、同原則に反する身体拘束であっても、
			\UTF{2460}先行の逮捕・勾留後に再度の身体拘束の必要性を示す新たな事情が生じたこと(事情変更)、
			\UTF{2461}「例外」としての逮捕の必要性は、通常の必要性よりも課徴されたものである必要があるから、
			再度の身体拘束が、それによって被る被疑者の不利益を考慮しても、なおやむを得ないといえるほどの高度の必要性があること(比較衡量)、
			\UTF{2462}逮捕・勾留の不当な蒸し返しに当たらないこと、という要件を満たせば、例外的に許容できる。
			
			もっとも、再勾留については刑訴法199条3項に相当する規定がないため、原則として認められないとする見解もある。
			しかし、同条は再逮捕の直接の根拠規定ではなく、逮捕と勾留は実体的要件を共有し、逮捕前置主義の下で密接に関連する一連の強制処分である。

			したがって、明文の規定がなくても、再逮捕が再勾留に発展することは当然に予定されていると解される。
			よって、再勾留であることや先行勾留の期間などを、\UTF{2461}の比較衡量の要件の中で考慮し、
			例外的に再勾留を認めるべき場合かどうかを判断すれば足りる。
			
		\subsection{別件逮捕・勾留}
			本件逮捕・勾留は、いわゆる別件逮捕・勾留にあたり違法ではないか。
			別件逮捕・勾留とは、本件逮捕・勾留について逮捕の要件が具備していないのに、
			その取調べのため、要件の具備している別件で逮捕・勾留することをいう。
			
			ここで、別件についての逮捕の要件・勾留を満たしている以上、その身体拘束は適法であるとの見解(別件基準説)がある。
			
			しかし、別件についての逮捕・勾留の要件を欠いている場合には「別件逮捕」を観念するまでもなく
			逮捕そのものが違法になるのは当然であって、
			捜査機関による権限の濫用により不当な身体拘束を行うという、
			別件逮捕・勾留の性質を軽視するものであって、妥当でない。
			
			したがって、別件について逮捕・勾留の要件が備わっていたとしても、
			別件での身柄拘束中の本件の取調べが令状主義の潜脱といえるようなものになった場合には、
			逮捕・勾留自体の適法性が否定されると解すべきである(本件基準説)。
			
			そもそも、逮捕・勾留の期間とは、被疑者の逃亡や証拠隠滅を防止しつつ、
			身柄拘束 の理由とされた被疑事実(別件)について、
			起訴・不起訴の判断を行うための捜査を進めるために設けられたものである。
			しかし、この期間が実際には主として本件の捜査に使われているような場合には、
			逮捕や勾留はもはや別件に基づくものとしての実体を失い、
			実質的には本件に基づく身柄拘束と評価されるべきである(実体喪失説)。
			
			そして、身柄拘束期間が主として本件捜査に利用されたか否かの判断は、
			\UTF{2460}別件・本件の取調べ状況、
			\UTF{2461}別件と本件の関連性の有無・程度、
			\UTF{2462}別件と本件との罪質の違い、
			\UTF{2463}別件について身柄拘束の必要性の程度、
			\UTF{2464}捜査官の意図・目的などの要素を考慮して行う。
		
			
	\section{捜索・押収・検証等}
	
		\subsection{令状による捜索・差押え}
			\subsubsection{差押え対象の特定}
			\subsubsection{令状の呈示}
			\subsubsection{必要な処分}
			\subsubsection{捜索の範囲}
			\subsubsection{捜索・差押えの際の写真撮影}
			\subsubsection{電磁的記録媒体の差押え}
			
			
		\subsection{令状によらない捜索・差押え}
			\subsubsection{時間的範囲}
				本件捜索差押えは、Xを逮捕するのに先行して無令状で行なわれたものである。そこで、本件捜索差押えは、
				逮捕に伴う捜索差押え(刑訴訟220条1項2号)として行なわれたものと考えられるが、適法か。
				
				「司法警察職員」は「被疑者を逮捕する場合」において「必要がある」ときに
				「逮捕の現場」で捜索差押え等を無令状で行うことができる(刑訴法220条1項2号、3項)。
				
				刑訴法が逮捕に伴う無令状の捜索差押え等を認めた趣旨は、
				被疑事実に関連する証拠が逮捕現場に存在する一般的な蓋然性があることを前提に、
				被逮捕者等による証拠隠滅の防止、逮捕執行者の身体の安全を確保する必要性に求められる(緊急処分説)。
				
				本問の捜索差押えは「司法警察員」である警察官Kらにより、「逮捕の現場」であるX宅にて行なわれている。
				また、Xが暴力団員であること、覚せい剤所持の現行犯で逮捕されたAが覚せい剤をXから購入したと供述したことから、
				X宅には覚せい剤が存在する蓋然性が認められる。
				加えて、覚せい剤は水に流すなどして証拠隠滅を図ることが容易であり、
				Xは、覚せい剤事犯の前科があるから、覚せい剤の取扱いに長けていると考えられる。
				したがって、証拠隠滅を防止する緊急の「必要がある」といえる。
				
				
				
				
			\subsubsection{場所的範囲}
			\subsubsection{対象物の範囲}
			
			
			
		
		
		\subsection{強制採尿}
			\subsubsection{強制採尿に必要な令状の種類}
				強制採尿は、体内に存在する尿を証拠物として強制的に採取する行為であるから、その行為は捜索・差押え(刑訴法218条1項)の性質を有する。
				
				したがって、強制採尿には捜索差押許可状(刑訴法218条1項、4項)を必要とするものと解すべきである。
				もっとも、強制採尿は、通常の捜索差押えとは異なり、身体に対する侵襲を伴うとともに、羞恥心を害し、屈辱感等の精神的打撃を与えるものである。
				そこで、強制採尿の実施にあたっては、被疑者の身体の安全と、その人格の保護のために十分な配慮がなされる必要がある。
				具体的には、医学的専門知識を有する医師によって、医学的に相当な方法を用いて行われることが必要であると解すべきである。
				その点では、強制採尿も、人の身体に対する検証としての身体検査(刑訴法218条1項後段)と同様の配慮を要する。
				
				したがって、捜索差押許可状も、本来身体検査令状についての規定である刑訴法218条6項を準用し、
				「強制採尿は、医師をして、医学的に相当と認められる方法により行わせなければならない」旨の条件を付した捜索差押許可状を請求すべきである。
				
			\subsubsection{採尿のための連行の可否}
				強制採尿は医師をして医学的に相当な方法によって行わせなければならないことが条件(刑訴法218条6項準用)とされる。
				この条件に従うためには、医師をして医学的に相当な方法によって採尿を行うのに適した設備のある病院等に対象者を連行して実施する必要があるので、
				「強制採尿令状の効力として、採尿に適する最寄りの場所まで被疑者を連行することができ」る。
				なぜなら、そのように解しないと、強制採尿令状の目的を達することができないし、
				このような場合に令状を発付する裁判官は「連行の当否を含めて審査し令状を発付したものとみられるからである。
				
		\subsection{強制採血}
			採血は注射器等を用いて被疑者の身体から直接血液を採取するため、被疑者の身体への侵襲を伴う。
			したがって、被疑者の身体に対する損傷を最小限にとどめ、その健康や安全に配慮するため、医学的な知識を有する医師によって行われる必要がある。
			すなわち、強制採血は、特別の専門知識に基づいて行われなければならず、
			高度の専門的知識・経験を事実に適用するという鑑定処分(刑訴法225条1項・168条1項)の性質を有するから、
			その実施にあたり鑑定処分許可状(刑訴法225条3項)が必要であると解すべきである。
			
			なお、いずれ老廃物として体外に排泄される尿と異なり、血液は生命・健康を維持に必要不可欠なものであり、体外に排泄されることを予定しておらず、
			むしろ身体の構成要素といえるから、尿のように証拠物としての性質を認めることは妥当ではない。
			したがって、物を捜索して占有を取得する行為であると観念するのは相当でないから、捜索差押許可状に基づいて行うことはできないと解すべきである。
			
			もっとも刑訴法225条4項は168条6項を準用するのみで、直接強制の規定である172条1項を準用せず、
			しかも、168条6項も直接強制の規定である139条を準用していないから、結局、鑑定処分許可状を用いただけでは、直接強制を行うことができなくなる。
			他方、検証としての身体検査を行うのであれば、直接強制が可能である(刑訴法222条1項後段、139条)。
			そこで、直接強制を行うための便宜として、身体検査令状(刑訴法218条1項後段)も併用すべきである。
			
			
\part{公訴}
	\section{訴因の特定}
		このような記載は、訴因の特定を要求する刑訴法256条3項に反しないか。
		
		刑訴法256条3項が訴因の特定を要求した趣旨は、刑事責任の追及原因として裁判所に対して審判を求める具体的な犯罪事実である訴因を特定することにより、
		審判の対象を明確にし、これを通じて防御の範囲を示すことにある。
		
		したがって、訴因を記載するに当たっては、\UTF{2460}被告人の行為が特定の構成要件を満たすか否かを判定するに足る程度の具体的事実を示すこと、\UTF{2461}他の犯罪との識別が可能なことが必要である。
		
		そして、これらのことを踏まえると、日時・場所・方法の記載は、基本的に訴因を特定する一手段と考えられるから、他の犯罪との識別に必要な限りで絶対的に必要である。
		
		一方、識別に必要といえない事項であっても、被告人の防御のため、証拠収集の限界等の特殊事情の下で、「できる限り具体的に」記載することが求められる。
		
	\section{訴因変更}
		\subsection{訴因変更の可否}
			裁判所は、訴因変更を許可することができるか。
			訴因の変更は、「公訴事実の同一性」を害しない限度において許される(刑訴法312条1項)。
			
			旧訴因から新訴因に訴因変更がなされうるとすると、被告人には、訴因変更が可能な範囲について訴追の可能性が生じているから、その範囲で手続の負担や有罪のリスクを負ったこととなる。
			その結果、一事不再理効が発生し(憲法39条、刑訴法337条1号)、また別訴によって処罰を求めることもできなくなる(二重起訴の禁止刑訴法338条3号、刑訴法339条1項5号)。
			
			このように、刑事訴訟法上、訴因変更が可能な範囲内(公訴事実の同一性の範囲内)では、1つの刑罰権に集約して処理することが想定されているといえる。したがって、新旧訴因が別訴で有罪となれば1個の刑罰権に関する二重起訴となるだろうという関係にあるとき、すなわち、法的に非両立の関係にあるときに「公訴事実の同一性」を認めるべきである。
			
			\noindent\fbox{窃盗から盗品の質入れあっせんへの訴因変更}
			
			「10月14日頃、静岡県において宿泊中のAから背広服を窃取した」との訴因を、「10月19日東京都内で同一の背広服をAから処分するよう依頼され、これを質入れした」との訴因への変更は可能か。
			
			\noindent\hangindent=1\zw →日時場所は離れているが、窃盗罪が成立すれば盗品等関与罪は成立せず、逆に、盗品等関与罪が成立するのであれば窃盗罪は成立していないことになるから、両罪は法的に非両立の関係にある。したがって、公訴事実の同一性が認められ、訴因変更を認めるべきである。
			
			\noindent\fbox{収賄から贈賄への訴因変更}
			
			「公務員Aと共謀の上、Aの職務上の不正行為に対する謝礼の趣旨で、Bから賄賂を収受した」との収賄の訴因を、「Bと共謀の上、同趣旨でAに対して賄賂を供与した」との贈賄の訴因への変更は可能か。
			
			\noindent\hangindent=1\zw →同一の金銭に関する賄賂に関与した被告人が賄賂を受ける側か、供与する側かのどちらか一方の罪しか成立せず、両罪は法的に非両立の関係にある。
			
			\noindent\fbox{覚せい剤使用罪の事例}
			
			使用日時・場所・方法に変更があり、併合罪が成立する可能性があるが、検察官の釈明により、最終行為か、最低1回の使用行為を起訴した趣旨であるとすれば、いずれか一方の罪しか成立せず、法的に非両立の関係にある。
			
			
			\noindent\fbox{相互にアリバイの関係にある場合}
			
			行為・結果ともに別個のもので共通性がなく、刑罰権の発動対象として完全に別事件であり、法的には両立する事件であるから、公訴事実の同一性は否定され、訴因変更はできない。
		\subsection{訴因変更の許否}
		\subsection{訴因変更の要否}
			本問では、訴因と裁判所の心証との間で、食い違いが生じている。
			刑事訴訟における審判の対象は訴因であり、訴因を逸脱した認定は違法である(刑訴法378条3号)。
			
			そこで、訴因と心証の間で、どのような事実の食い違いがあれば、訴因変更が必要かが問題となる。
		
			訴因とは、検察官が審判を請求する具体的な犯罪事実を記載したもの(事実記載説)であるから、訴因の拘束力は訴因の記載事実に求められる。
			したがって、事実に重要な差異が生じた場合に訴因変更が必要であると解される。
			では、どのような場合に「事実に重要な差異」が生じたといえるか。
			
			\UTF{2460}まず、訴因の機能である「審判対象の画定」の見地から、「罪となるべき事実」の特定に不可欠な事実について差異が生じた場合は、必ず訴因変更が必要となる。なぜなら、訴因に記載されたものとは別の審判対象を裁判所が設定し、認定してしまうと、検察官の訴因設定権限を害するからである。
			
			\UTF{2461}次に、\UTF{2460}に当たらない事実であっても、被告人の防御にとって重要な事項で、訴因に記載された事実と異なる認定をする場合、争点明確化による不意打ち防止の要請から、原則として訴因変更が必要である。
			
			\UTF{2462}もっとも、\UTF{2461}の場合でも、被告人の防御の具体的な状況に照らし、被告人に不意打ちとならず(他の手段で防御対象が明示されている場合)、
			かつ、認定事実が訴因記載事実と比べて被告人にとってより不利益でない場合(防御の利益が実質的に侵害されていない場合)には、なお不意打ち防止の要請に反しないので、例外的に訴因変更は必要でない。
	
	


		






\end{document}
