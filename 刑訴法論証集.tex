\documentclass[fontsize=11bp]{jlreq}
\usepackage{luatexja-fontspec}
\usepackage{luatexja-otf}
\setmainfont{PRyuminKSpProN-Light}
\setsansfont{PA1GothicStdN-Medium}
\setmainjfont{PRyuminKSpProN-Light}
\setsansjfont{PA1GothicStdN-Medium}

\begin{document}

\part{捜査法}
	\section{捜査の端緒}
		\subsection{職務質問}
			職務質問とは、不審事由のある者を「停止させて」「質問する」ことをいう(警職法2条1項)。
			職務質問の法的性質は任意処分であるから(警職法2条3項)、職務質問が強制処分に当たれば直ちに違法である。
			
			\subsubsection{強制処分の意義}
			強制処分には強制処分法定主義、令状主義の規律が及ぶ。
			そこで、強制処分とは、このような厳格な要件・手続によって統制を受けるにふさわしい処分であることを要するというべきである。
			具体的には、個人の意思に反して、身体、住居、財産等の重要な権利・利益に制約を加えて強制的に捜査目的を実現する行為など、
			特別の根拠規定がなければ許容することが相当でない手段を用いた処分をいうものと解すべきである。
			
			\subsubsection{任意処分の限界}
			任意処分であっても、一定の権利侵害の可能性が認められるから、情況のいかんを問わず常に許容されると解するのは妥当でなく、
			警職法1条1項および2項の比例原則により、
			当該手段を用いる必要性・緊急性と権利侵害の程度等を衡量し、
			具体的状況の下で相当と認められる限度でのみ許容されると解される。
			
			\subsection{職務質問のための停止}
			職務質問は任意処分であるが、有形力の行使はいかなる場合においても許容されないと解すると、警職法1条1項の目的を達成することができない。
			また、同法2条1項は「停止させて」質問することができると規定しており、有形力の行使を許容していると読める。
			
			したがって、一定程度の有形力の行使も許されると解する。もっとも、有形力の行使はあくまでも任意処分の枠の中で理解されるから、
			拒否の事由を残さないような実力の行使は許されないと解すべきである。
			
			また、任意手段であっても何らかの法益を侵害する可能性が認められるから、比例原則(警職法1条1項および2項)により、当該処分を行う必要性と緊急性を比較衡量した上、
			具体的状況の下で相当と認められる範囲でのみ許されると解する。
			
			\subsection{職務質問に伴う所持品検査}
			いわゆる所持品検査について警職法にはこれを許容する明文の規定は存在しない。
			しかし、侵害留保原則により権利侵害を伴う所持品検査をするには法的根拠が必要である。
			そこで、職務質問に伴う所持品検査の法的根拠をどこに求めるか、その許容性はどの程度かが問題となる。
			
			所持品検査は、職務質問と密接な関連性を有し、職務質問の効果をあげるうえでの必要性及び有効性が認められるから、
			警職法2条1項の職務質問の付随行為として行うことができる場合があると解される。
			
			\UTF{2460}所持品検査は、任意手段である職務質問の付随行為として許容されるのであるから、
			対象者の承諾を得て行うのが原則であるが、承諾のない所持品検査も「捜索に至らない程度の行為は、強制にわたらない限り」許容される場合がある。
			
			\UTF{2461}そして、任意手段であっても、所持品検査により憲法35条が保障する権利が害されるから、
			その限界は、所持品検査の必要性・緊急性と権利侵害の程度等を衡量し、具体的状況の下で相当と認められる限度でのみ許容される。
		
	
	\section{逮捕・勾留}
		\subsection{現行犯逮捕の要件}
			現行犯逮捕(刑訴法212条1項)を行うためには、
			\UTF{2460}犯罪と逮捕との時間的場所的接着性が認められることが必要である。
			また、現行犯逮捕が令状主義(憲法33条、刑訴法199条1項)の例外とされている趣旨から、
			\UTF{2461}逮捕者にとって犯罪と犯人が明白であることも要する。
			さらに、必要性のない身体拘束は正当化できないから、通常逮捕と同様、
			\UTF{2462}逮捕の必要性も要件となる。
		\subsection{準現行犯逮捕の要件}
			まず、被逮捕者が刑訴法212条2項各号のいずれかの要件に当たることが必要である。
			
			次に、Xが「罪を行ってから間がないと明らかに認められる」(同項柱書)か。
			ここで、「間がない」とは、犯罪行為と逮捕との時間的場所的接着性が認められることをいう。
			また、そもそも準現行犯人を現行犯人と見なして、無令状で逮捕できるのは、
			犯人であることが明らかであって、誤った逮捕などの人権侵害のおそれが小さいからである。
			そこで「明らか」とは、犯罪と犯人が逮捕者にとって明白であることをいうと解する。
			明白性の判断に当たっては、準現行犯逮捕の場合、ある程度の時間的・場所的な隔たりがあることは前提とされているから、
			逮捕者が直接知覚した事実のみならず、
			共犯者の供述や、既に得ている捜査情報等も補助的な判断資料として用いることが許される。
			
			また、準現行犯逮捕の場合、通常逮捕(刑訴法199条2項ただし書き)のように逮捕の必要性に関する明文の規定は存在しないが、
			身体拘束という重大な権利制約を伴う強制処分であるため、
			準現行犯逮捕も「明らかに逮捕の必要性がない」ときは許されないと解される。
			
		\subsection{再逮捕・再勾留禁止の原則、重複逮捕・勾留禁止の原則}
			\subsubsection{両原則の定義}
			
			再逮捕・再勾留禁止の原則とは、同一の被疑事実を基礎とする逮捕・勾留は1回しか許されないという原則をいう。
			
			重複逮捕・勾留禁止の原則とは、1つの被疑事実を複数に分割しその事実ごとに同時に複数の逮捕・勾留をすることは許されないという原則をいい、
			「1つの被疑事実」とは、実体法上一罪を意味する。
			実体法上一罪には1つの刑罰しか認められないから、刑事手続においても実体法上一罪を構成する事実を1つとして扱うべきであるからである。
			\subsubsection{思考プロセス}
			
			本件の逮捕・勾留は、その実体的要件が認められるから、手続を遵守していれば、
			刑訴法の明文の規定には反しない。
			
			しかし、逮捕・勾留は、原則として被疑事実ごとに一回しか許されない(一罪一逮捕一勾留の原則)。
			同一の被疑事実についても、逮捕・勾留を繰り返すことができるとすると、起訴前の身体拘束期間を制限した法の趣旨が没却されるからである。
			
			ここで、実体法上一罪を構成する被疑事実については1つの刑罰権しか認められないから、
			刑事手続上も実体法上一罪である被疑事実について同時に捜査する義務(同時処理義務)が課される。
			(また、実体法上一罪の関係になくとも、実質的に同一といえる関係にある被疑事実について
			\footnote{社会的事実としては、一連一体の事実であって、関係者も同一であり、必要とされる捜査の内容も大半が共通することなどを考慮する。}、
			同時処理可能性が肯定されれば、同時処理義務が発生する場合がある。)
			
			(被疑事実の同一性、同時処理可能性を確認して、同時処理義務を肯定したら)
			
			同時処理義務の認められる被疑事実につき、再度逮捕・勾留が行なわれた場合、再逮捕・再勾留禁止の原則に反し、直ちに違法となるか。
			
			捜査は流動的であるから、再度の身体拘束をする必要性が生じることは否定できない。
			また、刑訴法199条3項は、再逮捕がありうることを前提としていると解される。
			ここで、同原則の趣旨は、身体拘束の不当な蒸し返しを禁ずる点にあるから、それに当たらなければ、例外的に許容してよいと解される。
			
			したがって、同原則に反する身体拘束であっても、
			\UTF{2460}先行の逮捕・勾留後に再度の身体拘束の必要性を示す新たな事情が生じたこと(事情変更)、
			\UTF{2461}「例外」としての逮捕の必要性は、通常の必要性よりも課徴されたものである必要があるから、
			再度の身体拘束が、それによって被る被疑者の不利益を考慮しても、なおやむを得ないといえるほどの高度の必要性があること(比較衡量)、
			\UTF{2462}逮捕・勾留の不当な蒸し返しに当たらないこと、という要件を満たせば、例外的に許容できる。
			
			もっとも、再勾留については刑訴法199条3項に相当する規定がないことから、原則として許されないとの見解もあるが、
			同条は再逮捕の直接の根拠規定ではないから、
			
		
			
	\section{捜索・押収・検証等}
		\subsection{強制採尿}
			\subsubsection{強制採尿に必要な令状の種類}
				強制採尿は、体内に存在する尿を証拠物として強制的に採取する行為であるから、その行為は捜索・差押え(刑訴法218条1項)の性質を有する。
				
				したがって、強制採尿には捜索差押許可状(刑訴法218条1項、4項)を必要とするものと解すべきである。
				もっとも、強制採尿は、通常の捜索差押えとは異なり、身体に対する侵襲を伴うとともに、羞恥心を害し、屈辱感等の精神的打撃を与えるものである。
				そこで、強制採尿の実施にあたっては、被疑者の身体の安全と、その人格の保護のために十分な配慮がなされる必要がある。
				具体的には、医学的専門知識を有する医師によって、医学的に相当な方法を用いて行われることが必要であると解すべきである。
				その点では、強制採尿も、人の身体に対する検証としての身体検査(刑訴法218条1項後段)と同様の配慮を要する。
				
				したがって、捜索差押許可状も、本来身体検査令状についての規定である刑訴法218条6項を準用し、
				「強制採尿は、医師をして、医学的に相当と認められる方法により行わせなければならない」旨の条件を付した捜索差押許可状を請求すべきである。
				
			\subsubsection{採尿のための連行の可否}
				強制採尿は医師をして医学的に相当な方法によって行わせなければならないことが条件(刑訴法218条6項準用)とされる。
				この条件に従うためには、医師をして医学的に相当な方法によって採尿を行うのに適した設備のある病院等に対象者を連行して実施する必要があるので、
				「強制採尿令状の効力として、採尿に適する最寄りの場所まで被疑者を連行することができ」る。
				なぜなら、そのように解しないと、強制採尿令状の目的を達することができないし、
				このような場合に令状を発付する裁判官は「連行の当否を含めて審査し令状を発付したものとみられるからである。
				
		\subsection{強制採血}
			採血は注射器等を用いて被疑者の身体から直接血液を採取するため、被疑者の身体への侵襲を伴う。
			したがって、被疑者の身体に対する損傷を最小限にとどめ、その健康や安全に配慮するため、医学的な知識を有する医師によって行われる必要がある。
			すなわち、強制採血は、特別の専門知識に基づいて行われなければならず、
			高度の専門的知識・経験を事実に適用するという鑑定処分(刑訴法225条1項・168条1項)の性質を有するから、
			その実施にあたり鑑定処分許可状(刑訴法225条3項)が必要であると解すべきである。
			
			なお、いずれ老廃物として体外に排泄される尿と異なり、血液は生命・健康を維持に必要不可欠なものであり、体外に排泄されることを予定しておらず、
			むしろ身体の構成要素といえるから、尿のように証拠物としての性質を認めることは妥当ではない。
			したがって、物を捜索して占有を取得する行為であると観念するのは相当でないから、捜索差押許可状に基づいて行うことはできないと解すべきである。
			
			もっとも刑訴法225条4項は168条6項を準用するのみで、直接強制の規定である172条1項を準用せず、
			しかも、168条6項も直接強制の規定である139条を準用していないから、結局、鑑定処分許可状を用いただけでは、直接強制を行うことができなくなる。
			他方、検証としての身体検査を行うのであれば、直接強制が可能である(刑訴法222条1項後段、139条)。
			そこで、直接強制を行うための便宜として、身体検査令状(刑訴法218条1項後段)も併用すべきである。
			
			
		
	
		






\end{document}
